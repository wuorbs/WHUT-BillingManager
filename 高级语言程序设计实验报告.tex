\documentclass[UTF8]{ctexart}

\usepackage{geometry}
\usepackage{graphicx}
\usepackage{wrapfig}
\usepackage{array}
\usepackage[zh-cn]{datetime2}
\usepackage{enumerate}
\usepackage{enumitem}
\usepackage{listings}
\usepackage{fontspec}
\usepackage{xcolor}

\geometry{a4paper,left=3.18cm,right=3.18cm,top=2.54cm,bottom=2.54cm}
\setCJKmainfont{SimSun.ttc}[AutoFakeBold]
\linespread{1.7}
\newcolumntype{C}[1]{>{\centering\arraybackslash}m{#1}}
\newcolumntype{L}[1]{>{\raggedright\arraybackslash}m{#1}}
\CTEXsetup[format={\heiti\bfseries\zihao{-2}}]{section}
\CTEXsetup[format={\heiti\bfseries\zihao{3}}]{subsection}
\CTEXsetup[format={\heiti\bfseries\zihao{4}}]{subsubsection}
\newfontfamily{\consolas}{Consolas}
\definecolor{background_color}{RGB}{245,245,245}
\definecolor{str_color}{RGB}{34,139,34}
\lstset{
	language=c++,
	tabsize=4,
	breaklines,
	numbers=left,
	escapeinside=``,
	showspaces=false, 
	showstringspaces=false,
	extendedchars=false,
	xleftmargin=2em,
	basicstyle=\small\consolas,
	backgroundcolor=\color{background_color},
	keywordstyle=\color{blue},
	commentstyle=\color{gray},
	stringstyle=\color{str_color},
}

\begin{document}

	% 封面
	\zihao{4}
	%\pagestyle{empty}
	\raggedright
	\begin{tabular}{|C{1.59cm}|C{4.76cm}|}
		\hline
		{\bfseries 学号} & 0122210880104 \\
		\hline
	\end{tabular}\\
	{\zihao{2}\vspace{2\baselineskip}}
	\begin{center}
		\includegraphics[height=2.73cm,width=11.75cm]{whut-title.jpg} \\
		{\heiti\bfseries\zihao{2} 《程序综合设计实验》报告} \\
		\vspace{6\baselineskip}
		{\bfseries
			\begin{tabular}{C{2.68cm} C{6.64cm}}
				学\qquad 院 & 计算机与人工智能学院 \\
				\cline{2-2}
				专\qquad 业 & 计算机类 \\
				\cline{2-2}
				班\qquad 级 & 计算机类m2201 \\
				\cline{2-2}
				姓\qquad 名 & 吴明洲 \\
				\cline{2-2}
				指导教师 & 张蕊 \\
				\cline{2-2}
			\end{tabular}\\
		}
	\end{center}
	\vspace{3\baselineskip}
	\raggedleft
	\begin{tabular}{c c}
		日期 & \today \\
		\cline{2-2}
	\end{tabular}\\
	\raggedright

	% 目录
	\newpage
	\setcounter{page}{0}
	\thispagestyle{empty}
	\begin{center}
	\tableofcontents
	\end{center}
	
	%正文
	\newpage
	\pagestyle{plain}
	\zihao{-4}

	\section{实验目的}
	\qquad 通过迭代式开发,深入掌握 C/C++ 语言的文件、链表、结构体、动态内存管理等技术,开发实现一个计费管理软件。
	\begin{enumerate}[label=\arabic*),leftmargin=3em]
		\item 深入理解 C/C++ 语言的基本概念和基本原理,如数据类型等,熟练应用顺序选择和循环结构程序设计、函数、结构体、文件读写等基础技术。
		\item 掌握 C/C++ 语言的高级知识,如类、vector、链表等技术。
		\item 掌握模块化开发的具体实现方法,深入领会一些 C/C++ 程序设计实用开发和技巧。
		\item 了解迭代软件开发的一般过程,领会系统设计、系统实现以及系统测试的方法
	\end{enumerate}
	
	\section{系统功能与描述}
	\subsection{系统介绍}
	\qquad 本计费管理系统主要是模拟实现网吧收费的基本功能,提供消费功能、账户功能、管理员功能。
	\qquad 在网吧上机的用户首先在网吧进行开卡,注册一张上级卡。在往卡里充值后,每次上机时,系统根据上下机时间计算费用,从卡中扣除该费用。用户可以要求退回卡中的余额,也可以存放在卡中供下次上机使用。系统会记载用户每次消费、充值、退费的信息,供查询使用。 \\
	
	\subsection{功能结构}
	\qquad 系统功能结构图如下图所示。\\
	\includegraphics[width=15cm]{Func.png} \\
	
	\subsection{设计思路}
	\qquad 计费管理系统的所涉及到的数据存储在文本文件中。本系统有 3 个文本文件,分别是:
	\begin{itemize}[leftmargin=3em]
		\item card.txt 卡信息文件,存储所有上机卡
		\item billing.txt 消费、充值、退费信息等账单信息记录文件
		\item price.txt 价格信息文件
	\end{itemize}
	\begin{enumerate}[label=\arabic*),leftmargin=3em]
		\item 计费管理系统启动过程中要完成下列功能
		\begin{itemize}
			\item 从文本文件card.txt中读入所有卡信息,每张卡片信息存储在一个结构体变量中
			\item 从文本文件billing.txt中读入所有账单信息,每条账单信息存储在一个结构体变量中
			\item 用vector容器存储所有的卡片及账单信息
			\item 从文本文件price.txt中读入价格信息
		\end{itemize}
		\item 计费管理系统退出过程中要完成下列功能
		\begin{itemize}
			\item 保存对卡片信息的更新,即将所有卡片信息写入到文本文件card.txt
			\item 保存对账单信息的更新,即将所有账单信息写入到文本文件billing.txt
			\item 保存对价格信息的更新,即将价格信息写入到文本文件price.txt
		\end{itemize}
	\end{enumerate}
	
	\section{典型算法分析(完整源码请见光盘)}
	\subsection{二进制文件读写}
	\qquad 卡片信息和账单信息保存在容器vector中,vector容器是内存中的变量,当系统退出后,添加的数据就会消失,不能永久保存。要想实现永久保存,必须将信息写入文件中 \\
	\qquad 因为每张卡片信息或每条账单信息存储在结构体中,当我们使用常规的文件读写(文本文件读写)时,不便于我们处理数据。故在此考虑使用二进制文件读写操作,能够按结构体进行读写,便于我们操作数据 \\
	
	\subsubsection{文件流}
	\qquad 头文件 \#include<fstream> 提供了三个类,用来实现 C++ 对文件的操作:
	\begin{itemize}[leftmargin=3em]
		\item fstream(读写文件)
		\item ifstream(读文件)
		\item ofstream(写文件)
	\end{itemize}
	
	\subsubsection{文件打开方式和读写方式}
	{\bfseries 打开方式} \\
	\qquad 读:\lstinline{ifstream filename(char *buffer, ios::binary | ios::in)} \\
	\qquad 写:\lstinline{ofstream filename(char *buffer, ios::binary | ios::out)} \\
	\qquad 其中\lstinline{buffer}是一块内存地址,表示文本路径,用来存储或读取数据\lstinline{ios::binary}用于指定以二进制方式打开文件,\lstinline{ios::in}和\lstinline{ios::out}为缺省值,\lstinline{ios::out}打开文件的同时会截断文件内容。还有\lstinline{ios::app}不截断文件内容,只在文件末尾追加内容 \\
	{\bfseries 读写方式} \\
	\qquad 读:\lstinline{read(char *buffer, streamsize size)} \\
	\qquad 写:\lstinline{write(char *buffer, streamsize size)} \\
	\qquad 其中\lstinline{buffer}是一块内存地址,表示文本路径,用来存储或读取数据。\lstinline{size}表示要从缓存(\lstinline{buffer})中读出或写入的字符数
	
	\subsubsection{实现代码}
	\qquad 向文件中写入卡信息 \\
	\begin{lstlisting}
void CardVector::saveCard(const string &cardPATH) {
	ofstream cardfile(cardPATH, ios::out | ios::binary);
	if (!cardfile.is_open()) return;
	for (auto it: vec) cardfile.write((char *) &it, sizeof(Card));
	cardfile.close();
}
	\end{lstlisting}
	\qquad 更新指定的卡信息 \\
	\begin{lstlisting}
void CardVector::updateCard(const Card *p, const string &cardPATH, int CardIndex) {
	fstream cardfile(cardPATH, ios::in | ios::out);
	if (!cardfile.is_open()) return;
	cardfile.seekp(sizeof(Card) * CardIndex, ios::beg);
	cardfile.write((char *) p, sizeof(Card));
	cardfile.close();
}
	\end{lstlisting}
	\qquad 从文件中读取卡信息 \\
	\begin{lstlisting}
CardVector::CardVector(const string &cardPATH) {
	ifstream cardfile(cardPATH, ios::in | ios::binary);
	Card card{};
	if (!cardfile.is_open()) return;
	while (true) {
		cardfile.read((char *) &card, sizeof(Card));
		if (cardfile.eof())break;
		vec.push_back(card);
	}
	cardfile.close();
}
	\end{lstlisting}
	\qquad 对于账单信息和价格信息的操作类似
	
	\subsection{时间类型与字符串类型的互相转换}
	\subsubsection{与时间有关的知识}
	\qquad C++提供了与时间处理相关的一些函数和与时间处理有关的数据类型 time\_t 和 tm \\
	\qquad time\_t 是 64 位长整数,精确到秒,表示从 1970 年 1 月 1 日的零点开始到当前时间经过了多少秒 \\
	\qquad tm 是结构体类型,如下所示:
	\begin{lstlisting}
struct tm {
	int tm_sec; /*秒,0-59*/
	int tm_min; /*分钟,0-59*/
	int tm_hour; /*小时, 0-23*/
	int tm_mday; /*日,1-31*/
	int tm_mon; /*月,0-11*/ 
	int tm_year; /*年,从1900至今已经多少年*/
	int tm_wday; /*星期,从星期日算起,0-6*/
	int tm_yday; /*天数,0-365*/
	int tm_isdst; /*日光节约时间的旗标*/
};
	\end{lstlisting}
	\qquad 常见的时间函数有:\\
	\qquad \lstinline{time_t time(time_t *t)}\quad 取得从1970年1月1日的零点至今的秒数 \\
	\qquad \lstinline{time_t t = time(NULL)}\quad 获取本地时间 \\
	\qquad \lstinline{struct tm *localtime(const time_t *clock)}\quad 将长整数时间转换为结构体时间,从中得到年月日、星期、时分秒等信息 \\
	\qquad \lstinline{time_t mktime(struct tm *timeptr)}\quad 将tm结构的时间转换为长整数从1970年至今的秒数 \\
	\qquad 由于一般用户能够接受的时间是字符串“2017 年 3 月 8 日 15 时 30 分”这样的形式,因此我们需要编写自己的时间处理函数,将结构体 tm 中相关信息取出来组合成这样的字符
串 \\
	
	\subsubsection{实现代码}
	\begin{lstlisting}
// 定义函数timeToString,将时间转换成字符串格式
void timeToString(time_t t, char *Buf) {
    tm *timeinfo;
    timeinfo = localtime(&t);
    strftime(Buf, 20, "%Y-%m-%d %H:%M", timeinfo);
}

// 定义函数stringToTime,将字符串格式的时间转换成time_t类型
time_t stringToTime(char *Time) {
    tm tm1{};
    // 使用sscanf函数将字符串中的时间数据读入tm结构体
    sscanf(Time, "%d-%d-%d %d:%d", &tm1.tm_year, &tm1.tm_mon, &tm1.tm_mday, &tm1.tm_hour, &tm1.tm_min);
    tm1.tm_year -= 1900;
    tm1.tm_mon -= 1;
    tm1.tm_sec = 0;
    // 设置tm_isdst为-1,表示使用本地时区
    tm1.tm_isdst = -1;
    return mktime(&tm1);
}
	\end{lstlisting}
	
	\subsection{密码型文本输入}
	\qquad 系统在要求用户输入密码时,出于保护用户隐私和账户安全的考虑,一般将用户输入的密码显示为*号或·号 \\
	\qquad 运用到了\lstinline{_getch()}函数,其特点为从控制台读取一个字符,但不显示在屏幕上,因此我们在此基础上根据用户操作进行输出*号或退格即可
	\subsubsection{代码实现}
	\begin{lstlisting}
// 输入密码,以星号代替,并返回字符串
string CardVector::cinPwd() {
    char password[20];
    char ch;
    int i = 0;
    while ((ch = _getch()) != '\r') {
        if (ch == '\b') {
            if (i > 0) {
                i--;
                putchar('\b');
                putchar(' ');
                putchar('\b');
            }
        } else {
            password[i] = ch;
            i++;
            putchar('*');
        }
    }
    password[i] = '\0';
    return password;
}
	\end{lstlisting}	
	
	\subsection{模糊搜索(计算编辑距离)}
	\qquad 用户在进行查询卡信息时,可能因为不记得确切卡号或按错键而导致找不到卡,利用模糊搜索功能可在找不到卡时向用户提供最接近的卡号,以便于用户搜索 \\
	
	\subsubsection{计算编辑距离算法}
	\qquad 编辑距离(Edit Distance)最常用的定义就是Levenstein距离,是由俄国科学家Vladimir Levenshtein于1965年提出的,所以编辑距离一般又称Levenshtein距离。它主要作用是测量两个字符串的差异化程度,表示字符串a至少要经过多少个操作才能转换为字符串b,这里的操作包括三种:增加、删除、替换 \\
	\qquad 先从一个问题谈起:对于字符串"xyz"和"xcz",它们的最短距离是多少?我们从两个字符串的最后一个字符开始比较,它们都是'z',是相同的,我们可以不用做任何操作,此时二者的距离实际上等于"xy"和"xc"的距离,即d(xyz,xcz) = d(xy,xc)。也即是说,如果在比较的过程中,遇到了相同的字符,那么二者的距离是除了这个相同字符之外剩下字符的距离。即d(i,j) = d(i - 1,j-1) \\
	\qquad 接着,我们把问题拓展一下,最后一个字符不相同的情况:字符串A("xyzab")和字符串B("axyzc"),问至少经过多少步操作可以把A变成B \\
	\qquad 我们还是从两个字符串的最后一个字符来考察即'b'和'c'。显然二者不相同,那么我们有以下三种处理办法:
	\begin{enumerate}[label=(\arabic*),leftmargin=3em]
		\item 增加:在A末尾增加一个'c',那么A变成了"xyzabc",B仍然是"axyzc",由于此时末尾字符相同了,那么就变成了比较"xyzab"和"axyz"的距离,即d(xyzab,axyzc) = d(xyzab,axyz) + 1。可以写成d(i,j) = d(i,j - 1) + 1。表示下次比较的字符串B的长度减少了1,而加1表示当前进行了一次字符的操作
		\item 删除:删除A末尾的字符'b',考察A剩下的部分与B的距离。即d(xyzab,axyzc) = d(xyza,axyzc) + 1。可以写成d(i,j) = d(i - 1,j) + 1。表示下次比较的字符串A的长度减少了1
		\item 替换:把A末尾的字符替换成'c',这样就与B的末尾字符一样了,那么接下来就要考察出了末尾'c'部分的字符,即d(xyzab,axyzc) = d(xyza,axyz) + 1。写成d(i,j) = d(i -1,j-1) + 1表示字符串A和B的长度均减少了1。
	\end{enumerate}
	\qquad 按照以上思路,我们很容易写出下面的方程:\\
	\includegraphics[width=15cm]{LD.png} \\
	\qquad 可以使用递归或动态规划完成
	\subsubsection{代码实现(使用动态规划)}
	\begin{lstlisting}
// 计算两个字符串之间的编辑距离
int CardVector::editDistance(string str1, string str2) {
    int len1 = str1.size(), len2 = str2.size();
    vector<vector<int>> dp(len1 + 1, vector<int>(len2 + 1, 0));
    for (int i = 0; i <= len1; i++) dp[i][0] = i;
    for (int j = 0; j <= len2; j++) dp[0][j] = j;
    for (int i = 1; i <= len1; i++)
        for (int j = 1; j <= len2; j++)
            if (str1[i - 1] == str2[j - 1]) dp[i][j] = dp[i - 1][j - 1];
            else dp[i][j] = min(dp[i - 1][j], min(dp[i][j - 1], dp[i - 1][j - 1])) + 1;
    return dp[len1][len2];
}

// 返回最接近给定卡号的卡片
vector<Card> CardVector::getTopMatches(const string &CardName) {
    priority_queue<pair<int, int>, vector<pair<int, int>>, compareDistance> pq;
    vector<Card> res;
    for (int i = 0; i < vec.size(); i++) {
        int dist = editDistance(CardName, vec[i].CardName);
        pq.push({dist, i});
    }
    for (int i = 0; i < 5 && !pq.empty(); i++) {
        int idx = pq.top().second;
        res.push_back(vec[idx]);
        pq.pop();
    }
    return res;
}
	\end{lstlisting}
	\section{开发难点与体会}
	\begin{itemize}[leftmargin=3em]
		\item {\bfseries 难点一:}文件读写出现问题。在结构体中使用了 string 类型变量,写进文件时并未发现异常,但导致在初始化文件时读取文件异常,程序不能正常执行。原因是 string 类型变量实际上是一个指针,写入文件时,实际上写入的是 string 分配的动态地址而非 string 类型变量所指的内容 \\
		{\bfseries 解决方法:}把 string 类型变量改为用字符数组表示即可 \\
		{\bfseries 体会:}很难找到错误的原因,耗费大量时间
		\item {\bfseries 难点二:}函数重复声明问题。由于系统是分文件写的,当不同的文件互相引用时,很容易就出现函数的重复声明,导致系统不能运行 \\
		{\bfseries 解决方法:}理清不同文件的包含关系,在.h文件中使用\lstinline{#ifndef}或\lstinline{#pragma once} \\
		{\bfseries 体会:}调试难度极大,耗费心神
		\item {\bfseries 难点三:}不同的类之间方法的调用问题。不同的类之间可以实现友元类或友元函数以方便调用,但使用友元函数是总是出现不明原因的函数 \\
		{\bfseries 解决方法:}放弃使用友元,用老方法定义空对象再使用该对象调用方法 \\
		{\bfseries 体会:}C++中类的知识深奥复杂,任有很大一部分需要深入学习
	\end{itemize}
	
	\section{实验总结}
	\qquad 通过本次高级语言程序设计,使我对网吧计费管理系统有了进一步的认识和了解,也对开发系统的需求分析的步骤更加熟悉,并且能够利用Clion创建该系统,从而实现了对网吧计费的管理 \\
	\qquad 但是本系统只是一个初步的实现,而且,它还有一定的缺陷,比如还没有实现添加管理员服务,只能默认系统里固定的管理员,所以需要进一步分析以及进一步深入,使其更加的完善。在今后的学习过程中我会更加努力的学习这类知识,正确做出更好的系统
	
	% 评定表
	\newpage
	\pagestyle{empty}
	\centering
	\zihao{4}
	{\bfseries\zihao{2} 《程序综合设计实验》成绩评定表}\\
	{\zihao{2} \vspace{1\baselineskip}}
	\ 班级:计算机类m2201\qquad 姓名:吴明洲\qquad 学号:0122210880104 \\
	\begin{tabular}{|C{1cm}|L{7.62cm}|C{2.5cm}|C{2cm}|}
		\hline
		序号 & \multicolumn{1}{|c|}{评分项目} & 满分 & 实得分 \\
		\hline
		1 & 学习态度认真、遵守纪律 & 10 &  \\
		\hline
		2 & 迭代开发进度合理,提交结果正确 & 40 &  \\
		\hline
		3 & 代码规范、注释清晰、可读性好 & 10 &  \\
		\hline
		4 & 	软件功能完善、运行正确 & 20 &  \\
		\hline
		5 & 验收情况良好 & 10 &  \\
		\hline
		6 & 报告规范、描述清晰准确 & 10 &  \\
		\hline
		 & & 总评分 &  \\
		\hline
		\multicolumn{4}{|l|}{评语:} \\
		\multicolumn{4}{|l|}{} \\
		\multicolumn{4}{|l|}{} \\
		\multicolumn{4}{|l|}{} \\
		\multicolumn{4}{|l|}{} \\
		\multicolumn{4}{|l|}{} \\
		\multicolumn{4}{|l|}{} \\
		\multicolumn{4}{|l|}{} \\
		\multicolumn{4}{|l|}{} \\
		\multicolumn{4}{|l|}{} \\
		\hline
	\end{tabular}\\
	{\zihao{5}\vspace{2\baselineskip}}
	\raggedleft
	指导教师签名:\qquad \qquad \qquad \qquad \qquad \\
	\renewcommand{\today}{\number\year 年 \quad 月 \quad 日}
	\today
\end{document}